%%%%%%%%%%%%%%%%%%%%%%%%%%%%%%%%%%%%%%%%%%%%%%%%%%%%%%%%%%%%%%%%%%%%%%%%%%%%%
\subsection{Example}
%%%%%%%%%%%%%%%%%%%%%%%%%%%%%%%%%%%%%%%%%%%%%%%%%%%%%%%%%%%%%%%%%%%%%%%%%%%%%

Using data acquired from a T1 phantom at two flip angles,
$\alpha_1=60^\circ$ and $\alpha_2=120^\circ$, we compute the
multiplicative factor relative to the low flip angle using the
saturated double-angle method \citep{cun-pau-nay:saturated}.
  
<<doubleanglemethod>>=
sdam60 <- read.img("SDAM_ep2d_60deg_26slc.nii.gz")[,,1:26,]
dim(sdam60)
sdam120 <- read.img("SDAM_ep2d_120deg_26slc.nii.gz")[,,1:26,]
dim(sdam120)
sdam.image <- rowMeans(dam(sdam60, sdam120, 60), dims=3)
mask <- rowSums(sdam60, dims=3) > 500
dim(mask)
@ 
<<figure5-png,echo=FALSE,results=hide>>=
png("sdam.png", width=400, height=400)
@ 
<<figure5-code>>=
zi <- 10:13
w <- 5
SDAM <- read.img("SDAM_smooth.nii.gz")[,,1:26,,drop=TRUE]
par(mfrow=c(2,2), mar=rep(0,4))
for (z in zi) {
  image(sdam120[,,z,w], col=grey(0:128/128), xlab="", ylab="", axes=FALSE)
  image(ifelse(mask[,,z], SDAM[,,z], 0), zlim=c(.5,1.5),
        col=tim.colors(), add=TRUE)
}
@ 
<<figure5-dev.off,echo=FALSE,results=hide>>=
dev.off()
@ 

Figure~\ref{fig:sdam} is the estimated B1+ field (with isotropic
Gaussian smoothing) for a gel-based phantom containing a variety of T1
relaxation times.  The center of the phantom exhibits a flip angle
$>60^\circ$ while the flip angle rapidly becomes $<60^\circ$ when
moving away from the center in either the $x$, $y$ or $z$ dimensions.
Isotropic smoothing should be applied before using this field to
modify flip angles associated with additional acquisitions (e.g., in
the \pkg{AnalyzeFMRI} package).
  
\begin{figure}[!htbp]
  \centering
  \includegraphics*[width=.5\textwidth]{sdam.png}
  \caption{Estimated B1+ field (with isotropic Gaussian smoothing)
    using the saturated double-angle method.  The colors correspond to
    a multiplicative factor relative to the true flip angle
    ($60^\circ$).}
  \label{fig:sdam}
\end{figure}

Assuming the smoothed version of the B1+ field has been computed
(\code{SDAM}), multiple flip-angle acquisitions can be used to estimate
the T1 relaxation rate from the subject (or phantom).  The
multiplicative factor, derived from the saturated double-angle method,
is used to produce a spatially-varying flip-angle map and input into
the appropriate function.

<<t1estimation>>=
fnames <- c("fl3d_vibe-5deg","fl3d_vibe-10deg","fl3d_vibe-20deg",
            "fl3d_vibe-25deg","fl3d_vibe-15deg")
alpha <- c(5,10,20,25,15)
nangles <- length(alpha)
X <- Y <- 64
Z <- 26
flip <- fangles <- array(0, c(X,Y,Z,nangles))
for (w in 1:nangles) {
  vibe <- read.img(fnames[w])
  flip[,,,w] <- vibe
  fangles[,,,w] <- array(alpha[w], c(X,Y,Z))
}
TR <- 4.22 / 1000 # seconds
fanglesB1 <- fangles * array(SDAM, c(X,Y,Z,nangles))
zi <- 10:13
mask[,,(! 1:Z %in% zi)] <- FALSE
R1 <- R1.fast(flip, mask, fanglesB1, TR, verbose=TRUE)
@ 
<<figure6-png,echo=FALSE,results=hide>>=
png("t1_phantom.png", width=400, height=400)
@ 
<<figure6-code>>=
par(mfrow=c(2,2), mar=rep(0,4))
for (z in zi) {
  image(drop(vibe)[,,z], zlim=c(0,1024), col=grey(0:64/64),
        xlab="", ylab="", axes=FALSE)
  image(1/drop(R1$R10)[,,z], zlim=c(0,2.5), col=hotmetal(), add=TRUE)
}
@ 
<<figure6-dev.off,echo=FALSE,results=hide>>=
dev.off()
@ 

Figure~\ref{fig:t1-phantom} displays the quantitative T1 map for a
gel-based phantom using information from the estimated B1+ field.

\begin{figure}[!htbp]
  \centering
  \includegraphics*[width=.5\textwidth]{t1_phantom.png}
  \caption{Estimated T1 relaxation rates for the phantom data
  acquisition.  The colors range from 0-2.5 seconds.}
  \label{fig:t1-phantom}
\end{figure}

By defining regions of interest (ROIs) in 

<<FSLmask>>=
pmask <- read.img("t1_phantom_mask.nii.gz")
pmask[,,c(1:24,26),1] <- pmask[,,25,1] # repeat masked slice (#25) for all slices
dim(pmask)
@ 

\begin{figure}[!htbp]
\begin{center}
<<figure7,fig=TRUE,echo=TRUE>>=
T1 <- c(.484,.350,1.07,.648,.456,1.07,.660,1.543,1.543,.353)
par(mfrow=c(1,1), mar=c(5,4,4,2)+.1)
boxplot(split(1/drop(R1$R10), as.factor(drop(pmask)))[-1], 
        ylim=c(0,2.5), xlab="Region of Interest", ylab="T1 (seconds)")
points(1:10, T1, bg=rainbow(10), pch=21, cex=2)
@ 
\end{center}
\caption{Boxplots of the estimated T1 values for the gel-based
  phantom, grouped by user-specified regions of interest.  True T1
  values are plotted as colored circles for each distinct ROI.}
\label{fig:t1-phantom-boxplot}
\end{figure}

We may compare the ``true'' T1 values for each ROI with those obtained
from acquiring multiple flip angles with the application of B1
mapping.  Figure~\ref{fig:t1-phantom-boxplot} compares T1 estimates in
the 10 ROIs, defined by \code{pmask}, with the true T1 values (large
circles).  The first seven ROIs correspond to the cylinders that run
around the phantom, clockwise starting from approximately one o'clock.
The eighth and ninth ROIs are taken from the main compartment in the
phantom; ROI eight is drawn in the middle of the phantom while ROI
nine is drawn from the outside of the phantom.  The final ROI is taken
from the central cylinder embedded in the phantom.

\subsection{Contrast Agent Concentration}

The \code{CA.fast} function rearranges the assumed multidimensional
(2D or 3D) structure of the multiple flip-angle data into a single
matrix to take advantage of internal R functions instead of loops, and
called \code{E10.lm}.  Conversion of the dynamic signal intensities to
contrast agent concentration is performed via
\begin{equation*}
  [\text{Gd}] = \frac{1}{r_1}\left(\frac{1}{T_1} - \frac{1}{T_{10}}\right),
\end{equation*}
where $r_1$ is the spin-lattice relaxivity constant and $T_{10}$ is
the spin-lattice relaxation time in the absence of contrast media
\citep{buc-par:measuring}.  For computational reasons, we follow the
method of \cite{li-etal:improved}.

%%%%%%%%%%%%%%%%%%%%%%%%%%%%%%%%%%%%%%%%%%%%%%%%%%%%%%%%%%%%%%%%%%%%%%%%%%%%%
\subsection{Arterial Input Function}
%%%%%%%%%%%%%%%%%%%%%%%%%%%%%%%%%%%%%%%%%%%%%%%%%%%%%%%%%%%%%%%%%%%%%%%%%%%%%

Whereas quantitative PET studies routinely perform arterial
cannulation on the subject in order to characterize the arterial input
function (AIF), it has been common to use literature-based AIFs in the
DCE-MRI literature.  Examples include
\begin{equation*}
  C_p(t) = D \left( a_1 e^{-m_1t} + a_2 e^{-m_2t} \right),
\end{equation*}
where $D=0.1\,\text{mmol/kg}$, $a_1=3.99\,\text{kg/l}$,
$a_2=4.78\,\text{kg/l}$, $m_1=0.144\,\text{min}^{-1}$ and
$m_2=0.0111\,\text{min}^{-1}$
\citep{wei-lan-mut:pharmacokinetics,tof-ker:measurement}; or
$D=1.0\,\text{mmol/kg}$, $a_1=2.4\,\text{kg/l}$,
$a_2=0.62\,\text{kg/l}$, $m_1=3.0$ and $m_2=0.016$
\citep{fri-etal:measurement}.  There has been progress in measuring
the AIF using the dynamic acquisition and fitting a parametric model
to the observed data.  Recent models include \cite{par-etal:derived}
and \cite{ort-etal:efficient}.  \pkg{dcemriS4} has incorporated one of
these parametric models from \cite{ort-etal:efficient}
\begin{eqnarray*}
  C_p(t) &=& A_B t e^{-\mu_Bt} + A_G \left( e^{-\mu_Gt} +
  e^{-\mu_Bt}\right)%\\
  %C_p(t) &=& \left\{\begin{array}{ll}
  %a_B \left( 1 - \cos(\mu_Bt)\right) + a_B a_G f(t,\mu_G) &
  %0\leq{t}\leq{t_B}\\
  %a_B a_G f(t,\mu_G) e^{-\mu_G(t - t_B)} & t > t_B
  %\end{array}\right.
\end{eqnarray*}
which can be fitted to the observed data using nonlinear regression.
Using the AIF defined in \cite{buc:uncertainty}, we illustrate fitting
a parametric model to characterize observed data.  The
\code{orton.exp.lm} function provides this capability using a common
double-exponential parametric form.

<<buckley.aif>>=
data("buckley")
aifparams <- with(buckley, orton.exp.lm(time.min, input))
fit.aif <- with(aifparams, aif.orton.exp(buckley$time.min, AB, muB, AG, muG))
@
\begin{figure}[!htbp]
\begin{center}
<<figure8,fig=TRUE,echo=TRUE>>=
with(buckley, plot(time.min, input, type="l", lwd=2, xlab="Time (minutes)", 
                   ylab=""))
with(buckley, lines(time.min, fit.aif, lwd=2, col=2))
legend("topright", c("Simulated AIF", "Estimated AIF"), lwd=2, col=1:2)
@ 
\end{center}
\caption{Arterial input function (AIF) from \cite{buc:uncertainty} and
  the best parametric fit, using the exponential model from
  \cite{ort-etal:efficient}.}
\label{fig:fitted-aif}
\end{figure}

Figure~\ref{fig:fitted-aif} shows both the true AIF and the best
parametric description using a least-squares fitting criterion.

%%%%%%%%%%%%%%%%%%%%%%%%%%%%%%%%%%%%%%%%%%%%%%%%%%%%%%%%%%%%%%%%%%%%%%%%%%%%%
\section{Kinetic Parameter Estimation}
%%%%%%%%%%%%%%%%%%%%%%%%%%%%%%%%%%%%%%%%%%%%%%%%%%%%%%%%%%%%%%%%%%%%%%%%%%%%%

The standard Kety model \citep{ket:blood-tissue}, a single-compartment
model, or the extended Kety model, the standard Kety model with an
extra ``vascular'' term, form the collection of basic parametric
models one can apply using \pkg{dcemriS4}.  Regardless of which
parametric model is chosen for the biological system, the contrast
agent concentration curve at each voxel in the region of interest
(ROI) is approximated using the convolution of an arterial input
function (AIF) and the compartmental model; e.g.,
\begin{eqnarray*}
  C_t(t) & = & \ktrans \left[ C_p(t) \otimes e^{-\kep t} \right],\\
  C_t(t) & = & \vp C_P(t) + \ktrans \left[ C_p(t) \otimes e^{-\kep t}
    \right].
\end{eqnarray*}

Parameter estimation is achieved using one of two options in the
current version of this software:
\begin{itemize}
\item Non-linear regression using non-linear least squares
  (Levenburg-Marquardt optimization)
\item Bayesian estimation using Markov chain Monte Carlo (MCMC)
  \citep{sch-etal:TMI}
\end{itemize}
Least-square estimates of the kinetic parameters $\ktrans$ and $\kep$
(also $\vp$ for the extended Kety model) are provided in
\code{dcemriS4.lm} while the posterior median is provided in
\code{dcemriS4.bayes}.  When using Bayesian estimation all samples from
the joint posterior distribution are also provided, allowing one to
interrogate the empirical probability density function (PDF) of the
parameter estimates.

Using the simulated breast data from \cite{buc:uncertainty}, we
illustrate fitting the ``extended Kety'' model to the contrast agent
concentration curves using the exponential model for the AIF.  We use
non-linear regression to fit the data on an under-sampled subset (in
time) of
the simulated curves.

<<buckley.kinetic>>=
xi <- seq(5, 300, by=5)
img <- array(t(breast$data)[,xi], c(13,1,1,60))
time <- buckley$time.min[xi]
aif <- buckley$input[xi]
mask <- array(TRUE, dim(img)[1:3])
aifparams <- orton.exp.lm(time, aif)
fit <- dcemriS4.lm(img, time, mask, model="orton.exp",
                 aif="user", user=aifparams)
@ 
\begin{figure}[!htbp]
\begin{center}
<<figure9,fig=TRUE,echo=TRUE>>=
par(mfrow=c(4,4), mar=c(5,4,4,2)/1.25, mex=0.8)
for (x in 1:nrow(img)) {
  plot(time, img[x,1,1,], ylim=range(img), xlab="Time (minutes)",
       ylab="", main=paste("Series", x))
  kinparams <- with(fit, c(vp[x,1,1], ktrans[x,1,1], kep[x,1,1]))
  lines(time, model.orton.exp(time, aifparams[1:4], kinparams), 
        lwd=1.25, col=2)
}
@ 
\end{center}
\caption{Simulated signal intensity curves from Buckley (2002), for
  breast tissue, with the best parametric fit using an exponential
  model for the AIF and the ``extended Kety'' model.}
\label{fig:fitted-kinetic}
\end{figure}

Figure~\ref{fig:fitted-kinetic} displays the 13 unique simulated
curves along with the fitted curves from the compartmental model.
There is decent agreement between the observed and fitted values,
except for Series~6 which changes too rapidly in the beginning and
cannot be explained by the parametric model.

%%%%%%%%%%%%%%%%%%%%%%%%%%%%%%%%%%%%%%%%%%%%%%%%%%%%%%%%%%%%%%%%%%%%%%%%%%%%%
\section{Statistical Inference}
%%%%%%%%%%%%%%%%%%%%%%%%%%%%%%%%%%%%%%%%%%%%%%%%%%%%%%%%%%%%%%%%%%%%%%%%%%%%%

No specific support is provided for hypothesis testing in
\pkg{dcemriS4}.  We recommend one uses built-in facilities in
\proglang{R} to perform ANOVA (analysis of variance) or mixed-effects
models based on statistical summaries of the kinetic parameters over
the ROI per subject per visit.  An alternative to this traditional
approach is to analyze an entire study using a Bayesian hierarchical
model \citep{sch-etal:hierarchical}, available in the software project
\pkg{PILFER} (\url{pilfer.sourceforge.net}) .

One may also question the rationale for hypothesis testing in only one
kinetic parameter.  Preliminary work has been performed in looking at
the joint response to treatment of both $\ktrans$ and $\kep$ in
DCE-MRI by \cite{oco-etal:fPCA}.

\bibliography{dcemri}
\bibliographystyle{chicago}

\end{document}

